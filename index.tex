\documentclass[twoside,11pt]{article} 
\usepackage{jmlr2e} 
%\usepackage[colorlinks=true,linkcolor=blue,urlcolor=blue,citecolor=blue,breaklinks=true]{hyperref}
\usepackage{url}
\jmlrheading{1}{2011}{XX}{XX/11}{XXX}{Pedregosa, Varoquaux, Gramfort et al.}

% Short headings should be running head and authors last names

\usepackage{xcolor}
\usepackage[normalem]{ulem}
\usepackage{amsmath}

\newcommand{\GAEL}[1]{\textcolor{blue}{\uline{#1}}}
\newcommand{\FABIAN}[1]{\textcolor{red}{\uline{#1}}}


%%%%%%%%%%%%%%%%%%%%%%%%%%%%%%%%%%%%%%%%%%%%%%%%%%%%%%%%%%%%%%%%%%%%%%%%%%%%%%%%


\ShortHeadings{scikits.learn: machine learning in Python}{Pedregosa, Varoquaux, Gramfort et al.}
\firstpageno{1}


\begin{document}

\title{scikits.learn: machine learning in Python}


\author{\name Fabian Pedregosa \email fabian.pedregosa@inria.fr \\
        \name Ga\"el Varoquaux \email gael.varoquaux@normalesup.org  \\
        \name Alexandre Gramfort \email alexandre.gramfort@inria.fr \\
        \name Vincent Michel  \email vincent.michel@inria.fr \\
        \name Bertrand Thirion  \email bertrand.thirion@inria.fr \\
        \addr Parietal, INRIA Saclay / Neurospin, 
	    B\^at 145, CEA Saclay, 91191 Gif sur Yvette -- {\sc France}
        \AND
        \name Olivier Grisel \email olivier.grisel@ensta.fr \\
        \addr Nuxeo -- {\sc France} 
        \AND
        \name Mathieu Blondel \email mblondel@ai.cs.kobe-u.ac.jp \\
        \addr Kobe University, 1-1 Rokkodai, Nada, Kobe 657-8501 -- {\sc Japan} 
        \AND
        \name Peter Prettenhofer \email peter.prettenhofer@gmail.com \\
        \addr Bauhaus-Universit\"at Weimar -- {\sc Germany}
        \AND
        \name Ron Weiss \email ronweiss@gmail.com \\
        \addr MARL, NYU, New York -- {\sc USA}
        \AND
        \name Vincent Dubourg \email vincent.dubourg@gmail.com\\
        \addr  -- {\sc France}
        \AND
        \name Jake Vanderplas \email vanderplas@astro.washington.edu\\
        \addr University of Washington -- {\sc USA}
	\AND
        \name Alexandre Passos \email alexandre.tp@gmail.com \\
        \addr Unicamp,  Campinas -- {\sc Brazil}
        \AND
        \name David Cournapeau \email cournape@gmail.com \\
        \addr Kyoto -- {\sc Japan}
        \AND
        \name Matthieu Brucher \email matthieu.brucher@gmail.com \\
        \addr Total E\&P,  Pau -- {\sc France}
        \AND
        \name Matthieu Perrot \email matthieu.perrot@cea.fr\\
        \name \'Edouard Duchesnay \email edouard.duchesnay@cea.fr \\
        \addr LNAO / Neurospin, 
	    B\^at 145, CEA Saclay, 91191 Gif sur Yvette -- {\sc France}
}


\editor{?}

\maketitle

\begin{abstract} 
%
\emph{Scikits.learn} is a Python module integrating a wide range of
state-of-the-art machine learning algorithms for medium-scale supervised
and unsupervised problems. This package focuses on bringing machine
learning to non-specialists using a general-purpose high-level language.
Emphasis is put on ease of use, documentation, and API consistency
without sacrificing speed.
%
It has minimal dependencies and is distributed under the simplified BSD
license, encouraging its use in both academic and commercial settings.
Source code, binaries, and documentation can be downloaded from
\url{http://scikit-learn.sourceforge.net}.

\end{abstract}

\keywords{Python, supervised learning, unsupervised learning, model
selection}


%%  It offers a wide range of
%% methods such as Support Vector Machines, linear models (L1, L2
%% penalized), logistic regression, gaussian mixture models and more.


\section{Introduction}

The Python programming language is establishing itself as one of the
languages of choice for scientific computing. It is an appealing choice 
for algorithmic development and data processing in both academic and 
industrial settings thanks to its high-level interactive
nature and its maturing ecosystem of scientific libraries
\citep{cise2007,cise2011}.
In particular, as it is a general-purpose language, it enjoys success in
applied problems as well as amongst computer scientists.
%
{\sl Scikits.learn} harnesses this rich environment to provide
state-of-the-art machine learning tools to non-specialists, answering a
growing need for statistical data analysis in the software and web
industries, but also for basic research in non computer-science fields,
such as biology or physics.

The goal of the project is to provide a reference implementation of some
of the best known machine learning algorithms, while maintaining an
easy-to-use interface tightly integrated in the Python language.
\emph{Scikits.learn} differs from other learning toolboxes in Python for
various reasons: \emph{i)} it is distributed under the BSD license
\emph{ii)} it incorporates compiled code for efficiency --unlike MDP
\citep{zito2008} and pybrain \citep{schaul2010}-- \emph{iii)} it depends
only on numpy and scipy --unlike pymvpa \citep{hanke2009}, we avoid depending
on R for distribution reasons-- \emph{iv)} it focuses on imperative
programming and not data-flow frameworks --pybrain uses such a framework
for building artificial neural networks.

While the package is mostly coded in Python, it incorporates the C++
libraries LibSVM \citep{chang2001} and LibLinear \citep{fan2008} that
provide reference implementations of SVMs and generalized linear models
with compatible licenses.
%
Binary packages are available on a rich set of platforms including
Windows and any POSIX platforms. Furthermore, it is distributed as part
of major open source distributions such as Ubuntu, Debian, Mandriva,
NetBSD or Macports and in commercial distributions such as ``Enthought
Python Distribution'' thanks to its liberal license.

%%The library is structured in submodules, each of which covers a family
%%of related algorithms. This way, the svm module support vector
%%machine-related algorithms of classification (SVC), regression (SVR)
%%and unsupervised learning (OneClassSVM).
%%
%%% Mention 'flat is better than nested'

%%% Mention the quality of our SVM bindings


\section {Project vision}

\noindent{\bf Code quality}
%
Rather than aiming for many features, our goal is to provide solid
implementations. The quality of the code base is ensured with unitary
tests: as of release 0.6, test coverage is 78\%. We follow the Python
coding style guidelines, as well as the numpy style for documentation. We
employ static analysis tools such as {\tt pyflakes} and {\tt pep8}.
Finally, we strive to have consistent naming for the various functions
and parameters.

\smallskip \noindent{\bf BSD licensing}
%
Most of the Python ecosystem is licensed with non copyleft licenses such
as the BSD license. While this licensing scheme is beneficial for adoption
outside the academic world via the use of these tools in commercial
projects, it does impose some restrictions: we cannot use some existing
scientific code, such as the GSL (GNU Scientific Library).

\smallskip \noindent{\bf Bare-bone design and API}
%
For a low barrier of entry, we avoid framework code and keep the number
of different objects to a minimum. In particular, we use as a dataset
object the numpy array \citep{Vanderwalt2011} that is common to the
scientific Python ecosystem.

\smallskip
\noindent{\bf Community-driven development}
%
We base our development on collaborative tools such as git, github and
public mailing lists. External contributions are welcome and
encouraged via developer manual and API guidelines.

\smallskip \noindent{\bf Documentation}
%
We believe that documentation is a key component of software.
\emph{Scikits.learn} provides a $\sim$300 page user guide including
narrative documentation, class reference, tutorial, installation
instructions, as well as more than 60 examples, some of them featuring
real-world applications. We try to minimize the amount of
machine-learning specific terms used, while maintaining precision with
regards to the algorithms employed.


\section{Underlying technologies}

%\emph{Scikits.learn} builds upon a powerful stack of existing libraries:

%XXX: Shorten the above

\noindent{\bf Numpy}:
%
the base data structure for representing
information. The data for supervised or unsupervised learning is
presented as one, or a few, numpy arrays, thus integrating seamlessly
with other scientific Python libraries. Numpy's view-based memory 
model limits copies, even when binding with compiled code. It also 
provides basic arithmetic operations. 
%This choice strives for easy of use by using plain numpy
%arrays instead of specialized data structures and limiting framework
%code. Our objects work directly with NumPy arrays, providing a
%flexible way to communicate with other packages.

\smallskip
\noindent{\bf Scipy}:
%
efficient algorithms for linear algebra, sparse matrix structure, special
functions and basic statistical functions. {\sl Scipy} has binding for
many Fortran-based standard numerical packages. This is important for
ease of install and portability, as providing binaries binding Fortran
code can prove challenging on various platforms. 

\smallskip
\noindent{\bf Cython}:
%
generation of C code seamlessly inserted in Python. Cython allows to
reach the performance of compiled languages as C while retaining
Python-like syntax and high-level operations. Also, Cython is used to
easily bind compiled libraries, eliminating most of the boilerplate code
associated with Python/C extensions.

%\smallskip
%\noindent{\bf Matplotlib}:
%%
%visualization and plotting. Matplotlib is used only in the examples.

\section{Code design}

\noindent{\bf Objects specified by interface, not by inheritance}
%
To facilitate the use of external objects with \emph{scikits.learn},
inheritance is not enforced at any stage; instead, we rely on some simple
conventions. The central object is an {\tt estimator}, that implements a
{\tt fit} method, accepting as arguments an input data array and,
optionally, a target array for supervised problems. Supervised estimators,
such as SVCs, can implement a {\tt predict} method. Some estimators,
that we call {\tt transformers}, \emph{e.g.} the PCA, implement a {\tt
transform} method, returning modified input data.
%
Estimators may also provide a {\tt score} method, evaluating the goodness
of fit of an estimator. It may be a log-likelihood, or the opposite of a
loss function, but must always increase with goodness of fit.
%
The only other important object is the \emph{cross-validation iterator}:
it is an iterable providing a pair of train and test indices to split input
data. It may be a standard list, or a Python generator dynamically
creating these indices. Examples comprise K-fold, leave one out, or
stratified schemes. 


\smallskip \noindent{\bf Model selection}
%
\emph{Scikits.learn} can evaluate the performance of an estimator or set
its parameters by cross-validation, optionally distributing the
computation to several cores. For this, it provides a {\tt GridSearchCV}
object, where the ``CV'' stands for ``cross-validated'', that wraps a
given estimator. During the call to {\tt fit}, it selects the parameters
on a specified parameter grid, maximizing a score function that defaults
to the {\tt score} method of the original estimator. The calls to {\tt
predict}, {\tt score}, or {\tt transform} are then delegated to the tuned
estimator. This object can thus be used transparently as any other
estimator. Such cross validation can be made more efficient for certain
estimators by exploiting their specific properties, such as warm restarts
or regularization paths. This is done via special cross-validated
objects, such as the {\tt LassoCV}. Finally, a {\tt Pipeline} object can
combine several {\tt transformers} and a final estimator to create a
combined estimator applying \emph{e.g.} dimension reduction before the
fit. {\tt Pipeline} behave as standard estimators, and can be thus used
with the {\tt GridSearchCV} to tune the parameters of all steps together.

\section{High-level yet efficient: some trade offs}

%Special care has been take on algorithmic efficiency, producing
%algorithms that are ofter faster than the ones found in compiled
%libraries.

While \emph{scikits.learn} focuses on ease of the use, and it is
mostly written in a high level language, care has been taken on
computational efficiency. In table \ref{tab:comparisons}, we compare
computation time for a few algorithms implemented in the major machine
learning toolkits accessible in Python. We use the Madelon data
set \citep{Guyon2004}, 4400 instances and 500 attributes,
that can be used in supervised
and unsupervised settings and is quite large, but small enough for most
algorithms to run.
% which was part of the NIPS 2003 feature selection challenge. It is an
% artificial dataset
%containing data points grouped in 32 clusters placed on the vertices of
%a five dimensional hypercube and randomly labeled +1 or -1.
%
%
In the following, we discuss some aspects of these results, to illustrate
the engineering trade offs related to performance. 

%{{{----------------------------------------------------------------------------
\begin{table}[t]
\small
\hspace*{.03\linewidth}%
%\begin{minipage}{1.04\linewidth}
\begin{tabular}{l c c c c c c}
\hline\hline %inserts double horizontal lines 
 & scikits.learn & mlpy & pybrain & pymvpa &  mdp & shogun \\ [0.5ex]
\hline
Support Vector Classification & 9.44 & 16.8 & 17.5 & 26.1 & 52.4 & {\bf 8.68} \\
Lasso (LARS) & {\bf 1.40} & 113.   & - &  37.5 & - & - \\
Elastic Net & {\bf 0.42} & 73.1 & -  &  6.53  & -  & - \\
k-Nearest Neighbors & {\bf 1.64} & 4.29 & - &  1.73 & {\bf 1.64} & 4.14 \\
PCA (9 components) & {\bf 0.18} & 3.01  & 2.77  & - & 0.50  & - \\
k-Means (9 clusters) & 1.00 &  {\bf 0.64} & $\star$ & -  & 27.8 & 0.76 \\
License &  BSD & GPL & BSD  &  BSD  & BSD  & GPL \\
\hline
\end{tabular}

-: Not implemented. \hfill
$\star$: Not converging after 1 hour iteration.

\vspace*{-1.5ex}
%\end{minipage}
\caption{\small
Time in seconds on the Madelon dataset for various machine learning libraries exposed in Python:
MLPy \citep{albanese2008}, PyBrain \citep{schaul2010}, pymvpa
\citep{hanke2009}, MDP \citep{zito2008} and Shogun
\citep{sonnenburg2010}. Code for running the
benchmarks can be retrieved from {\tt http://github.com/scikit-learn}.
\vspace*{-1.5em}\label{tab:comparisons}
}
\end{table}
%----------------------------------------------------------------------------}}}

\smallskip \noindent{\bf SVM}:
%
While all packages compared use libsvm in the background, good
performance of \emph{scikits.learn} can be explained by two factors.
First, our bindings have up to 40\% less overhead than the original
libsvm Python bindings: they are implemented with Cython's numpy support
to avoid memory copies. Second, we patch libsvm to work efficiency
on dense data, providing less memory footprint, better usage of memory
alignment and pipelining capabilities of modern processors. This patched
version also provides unique features, such as working without copies on
sparse data or setting weights for individual samples.


\smallskip \noindent{\bf LARS}:
%
Iteratively refining the residuals instead of recomputing them gives
performance gains of 2x to 10x over the reference R implementation
\citep{LARS}. {\sl Pymvpa} uses this implementation via the Rpy R
bindings and pays a heavy price to memory copies.


\smallskip \noindent{\bf Elastic Net}:
%
We benchmarked coordinate descent implementations of Elastic Net.  In
\emph{scikits.learn}, it is coded in Cython and uses low-level BLAS
routines for fast array operations. It achieves the same order of
performance as the highly optimized Fortran version \emph{glmnet}
\citep{friedman2010} on medium-scale problems, but performance on very
large problems will be limited as we do not use the KKT conditions to
define an active set.

\smallskip
\noindent{\bf kNN}:
%
The k-nearest neighbors is achieved by constructing a ball
tree \citep{omohundro1989} of the samples, but defaults to a more
efficient brute force search in large dimensions.

\smallskip \noindent{\bf PCA}:
%
For medium to large datasets, \emph{scikits.learn} provides an
implementation of a truncated PCA based on random projections
\citep{rokhlin2009}. On medium-scale problems, incomplete eigenvalue
decomposition, as used by MDP, also performs very well.

\smallskip 
\noindent{\bf k-Means}:
%
The performance of \emph{scikits.learn}'s k-Means is limited by numpy's
array operations that operate multiple passes over data, as it is
implemented in pure Python.

\section{Conclusion}

\emph{Scikits.learn} exposes a wide variety of machine learning
algorithms, supervised and unsupervised, with a consistent,
task-oriented interface, thus enabling easy comparison of methods for a
given application.
%
As it relies on the scientific Python ecosystem, it can easily be
integrated in applications outside the traditional range of statistical
data analysis. Importantly, the algorithms, implemented in a high-level
language, can be used as building blocks for approaches dedicated to
specific use cases, \emph{e.g.} in medical imaging \citep{Michel2011}.
%
Future work include the addition of \emph{on line} learning, to scale on
large datasets.

\bibliography{scikit}

\end{document}
